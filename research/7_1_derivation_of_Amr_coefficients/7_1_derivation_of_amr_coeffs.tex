\documentclass[12pt, letterpaper]{amsart}
\usepackage[left=1in,right=1in,bottom=1in,top=1in]{geometry}
\usepackage{amsfonts}
\usepackage{amsmath, amssymb}
\usepackage[font=small,labelfont=bf]{caption}
\usepackage[pdfpagelabels,hyperindex,colorlinks=true,linkcolor=blue,urlcolor=magenta,citecolor=green]{hyperref}
\usepackage{amsthm}
\usepackage{float}
\usepackage{mathrsfs}
\usepackage{colonequals}
\usepackage{natbib}

\hypersetup{
pdftitle={Manuscript title},
pdfsubject={Mathematics},
pdfauthor={Petro Kolosov},
pdfkeywords={}
}


\newtheorem{thm}{Theorem}[section]
\newtheorem{cor}[thm]{Corollary}
\newtheorem{prop}[thm]{Proposition}
\newtheorem{lem}[thm]{Lemma}
\newtheorem{conj}[thm]{Conjecture}
\newtheorem{quest}[thm]{Question}
\newtheorem{ppty}[thm]{Property}
\newtheorem{ppties}[thm]{Properties}
\newtheorem{claim}[thm]{Claim}

\theoremstyle{definition}
\newtheorem{defn}[thm]{Definition}
\newtheorem{defns}[thm]{Definitions}
\newtheorem{con}[thm]{Construction}
\newtheorem{exmp}[thm]{Example}
\newtheorem{exmps}[thm]{Examples}
\newtheorem{notn}[thm]{Notation}
\newtheorem{notns}[thm]{Notations}
\newtheorem{addm}[thm]{Addendum}
\newtheorem{exer}[thm]{Exercise}
\newtheorem{limit}[thm]{Limitation}

\theoremstyle{remark}
\newtheorem{rem}[thm]{Remark}
\newtheorem{rems}[thm]{Remarks}
\newtheorem{warn}[thm]{Warning}
\newtheorem{sch}[thm]{Scholium}

\newenvironment{myitemize}
{ \begin{itemize}
    \setlength{\itemsep}{4pt}
    \setlength{\parskip}{4pt}
    \setlength{\parsep}{4pt}     }
{ \end{itemize}                  }

\makeatletter
\let\c@equation\c@thm
\raggedbottom
\makeatother
\numberwithin{equation}{section}
%--------Meta Data: Fill in your info------
%&#1043;&#1086;&#1089;&#1087;&#1086;&#1076;&#1080;, &#1048;&#1080;&#1089;&#1091;&#1089;&#1077; &#1061;&#1088;&#1080;&#1089;&#1090;&#1077;, &#1057;&#1099;&#1085;&#1077; &#1041;&#1086;&#1078;&#1080;&#1081;, &#1084;&#1086;&#1083;&#1080;&#1090;&#1074;&#1072;&#1084;&#1080; &#1055;&#1088;&#1077;&#1095;&#1080;&#1089;&#1090;&#1099;&#1103; &#1058;&#1074;&#1086;&#1077;&#1103; &#1052;&#1072;&#1090;&#1077;&#1088;&#1080; &#1080; &#1042;&#1089;&#1077;&#1093; &#1057;&#1074;&#1103;&#1090;&#1099;&#1093; &#1058;&#1074;&#1086;&#1080;&#1093;, &#1087;&#1086;&#1084;&#1080;&#1083;&#1091;&#1081; &#1085;&#1072;&#1089;. &#1040;&#1084;&#1080;&#1085;&#1100;.
\begin{document}

\section{Derivation of A(m,r) coeffs}
Assuming that 
\begin{equation*}
n^{2m+1} = \mathbf{P}^{m}_{n}(n) = \sum\limits_{k}\mathbf{A}_{m,k}\mathbf{Q}^{k}_{n}(n), \quad m\geq 0.
\end{equation*}
The coefficients $A_{m,r}$ could be evaluated expanding $\sum_{k=0}^{n-1}k^r(n-k)^r$ and using Faulhaber's formula $\sum _{k=1}^{n}k^{p}=\tfrac{1}{p+1}\sum _{j=0}^{p}{p+1 \choose j}B_{j}n^{p+1-j}$, we get
\begin{equation}\label{proof1}
\begin{split}
&\sum_{k=0}^{n-1}k^r(n-k)^r\\
&=\sum_{k=0}^{n-1} k^r \sum_{j} (-1)^j\binom{r}{j} n^{r-j}k^{j}=\sum_{j} (-1)^j\binom{r}{j} n^{r-j}\left(\sum_{k=0}^{n-1}k^{r+j}\right)\\
&=\sum_{j} \binom{r}{j} n^{r-j}\frac{(-1)^j}{r+j+1}\left[\sum_{s}\binom{r+j+1}{s}B_{s}n^{r+j+1-s}-B_{r+j+1}\right]\\
&=\sum_{j,s}\binom{r}{j}\frac{(-1)^j}{r+j+1}\binom{r+j+1}{s}B_{s}n^{2r+1-s}-\sum_{j} \binom{r}{j}\frac{(-1)^j}{r+j+1}B_{r+j+1}n^{r-j}\\
&=\sum_{s}\underbrace{\sum_{j}\binom{r}{j}\frac{(-1)^j}{r+j+1}\binom{r+j+1}{s}}_{S(r)}B_{s}n^{2r+1-s}-\sum_{j} \binom{r}{j}\frac{(-1)^j}{r+j+1}B_{r+j+1}n^{r-j}
\end{split}
\end{equation}
where $B_s$ are Bernoulli numbers and $B_1=\tfrac12$.
Now, we notice that
\begin{equation*}
S(r)=\sum_{j} \binom{r}{j}\frac{(-1)^j}{r+j+1}\binom{r+j+1}{s}
=\begin{cases}
\frac{1}{(2r+1)\binom{2r}r}, & \text{if } s=0;\\
\frac{(-1)^r}{s}\binom{r}{2r-s+1}, & \text{if } s>0.
\end{cases}
\end{equation*}
In particular, the last sum is zero for $0<s\leq r$. Therefore, expression (\hyperref[proof1]{3.1}) takes the form
\begin{equation*}
\begin{split}
\sum_{k=0}^{n-1}k^r(n-k)^r
&=\frac{1}{(2r+1)\binom{2r}r}n^{2r+1}+\underbrace{\sum_{s\geq 1}\frac{(-1)^r}{s}\binom{r}{2r-s+1}B_{s}n^{2r+1-s}}_{(\star)}\\
&-\underbrace{\sum_{j} \binom{r}{j}\frac{(-1)^j}{r+j+1}B_{r+j+1}n^{r-j}}_{(\diamond)}
\end{split}
\end{equation*}
Hence, introducing $\ell=2r+1-s$ to $(\star)$ and $\ell=r-j$ to $(\diamond)$, we get
\begin{equation*}
\begin{split}
\sum_{k=0}^{n-1}k^r(n-k)^r
&=\frac{1}{(2r+1)\binom{2r}r}n^{2r+1}+\sum_{\ell}\frac{(-1)^r}{2r+1-\ell}\binom{r}{\ell}B_{2r+1-\ell}n^{\ell}\\
&-\sum_{\ell} \binom{r}{\ell}\frac{(-1)^{j-\ell}}{2r+1-\ell}B_{2r+1-\ell}n^{\ell}\\
&=\frac{1}{(2r+1)\binom{2r}r}n^{2r+1}+2\sum_{\mathrm{odd} \ \ell}\frac{(-1)^r}{2r+1-\ell}\binom{r}{\ell}B_{2r+1-\ell}n^{\ell}
\end{split}
\end{equation*}
Using the definition of $A_{m,r}$ coefficients, we obtain the following identity for polynomials in $n$
\begin{equation}\label{proof2}
\sum_{r}A_{m,r}\frac{1}{(2r+1)\binom{2r}r}n^{2r+1}+2\sum_{r, \ \mathrm{odd} \ \ell}A_{m,r}\frac{(-1)^r}{2r+1-\ell}\binom{r}{\ell}B_{2r+1-\ell}n^{\ell}\equiv n^{2m+1}
\end{equation}
Taking the coefficient of $n^{2m+1}$ in (\hyperref[proof2]{3.2}) we get $A_{m,m}=(2m+1)\binom{2m}m$ and taking the coefficient of $n^{2d+1}$ for an integer $d$ in the range $m/2 \leq d < m$, we get $A_{m,d}=0$. Taking the coefficient of $n^{2d+1}$ for $d$ in the range $m/4 \leq d < m/2$, we get
\begin{equation*}
A_{m,d} \frac{1}{(2d+1)\binom{2d}{d}} + 2 (2m+1) \binom{2m}{m} \binom{m}{2d+1} \frac{(-1)^m}{2m-2d} B_{2m-2d} = 0,
\end{equation*}
i.e,
\begin{equation*}
A_{m,d} = (-1)^{m-1} \frac{(2m+1)!}{d!d!m!(m-2d-1)!}\frac{1}{m-d} B_{2m-2d}.
\end{equation*}
Continue similarly, we can express $A_{m,d}$ for each integer $d$ in range $m/2^{s+1}\leq d< m/2^s$ (iterating consecutively $s=1,2...$) via previously determined values of $A_{m,j}$ as follows
\begin{equation*}
A_{m,d} = (2d+1)\binom{2d}{d} \sum_{j\geq 2d+1} A_{m,j} \binom{j}{2d+1} \frac{(-1)^{j-1}}{j-d} B_{2j-2d}.
\end{equation*}
Thus, for every $(n,m)\in\mathbb{N}$ holds
\begin{equation*}
n^{2m+1}=\sum_{r=0}^{m}A_{m,r}\sum_{k=0}^{n-1}k^r(n-k)^r.
\end{equation*}
\end{document}
