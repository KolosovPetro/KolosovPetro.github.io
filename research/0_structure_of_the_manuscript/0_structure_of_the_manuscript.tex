\documentclass[12pt, letterpaper]{amsart}
\usepackage[left=1in,right=1in,bottom=1in,top=1in]{geometry}
\usepackage{amsfonts}
\usepackage{amsmath, amssymb}
\usepackage[font=small,labelfont=bf]{caption}
\usepackage[pdfpagelabels,hyperindex,colorlinks=true,linkcolor=blue,urlcolor=magenta,citecolor=green]{hyperref}
\usepackage{amsthm}
\usepackage{float}
\usepackage{mathrsfs}
\usepackage{colonequals}
\usepackage{natbib}

\hypersetup{
pdftitle={0_structure_of_the_manuscript},
pdfsubject={Mathematics},
pdfauthor={Petro Kolosov},
pdfkeywords={}
}


\newtheorem{thm}{Theorem}[section]
\newtheorem{cor}[thm]{Corollary}
\newtheorem{prop}[thm]{Proposition}
\newtheorem{lem}[thm]{Lemma}
\newtheorem{conj}[thm]{Conjecture}
\newtheorem{quest}[thm]{Question}
\newtheorem{ppty}[thm]{Property}
\newtheorem{ppties}[thm]{Properties}
\newtheorem{claim}[thm]{Claim}

\theoremstyle{definition}
\newtheorem{defn}[thm]{Definition}
\newtheorem{defns}[thm]{Definitions}
\newtheorem{con}[thm]{Construction}
\newtheorem{exmp}[thm]{Example}
\newtheorem{exmps}[thm]{Examples}
\newtheorem{notn}[thm]{Notation}
\newtheorem{notns}[thm]{Notations}
\newtheorem{addm}[thm]{Addendum}
\newtheorem{exer}[thm]{Exercise}
\newtheorem{limit}[thm]{Limitation}

\theoremstyle{remark}
\newtheorem{rem}[thm]{Remark}
\newtheorem{rems}[thm]{Remarks}
\newtheorem{warn}[thm]{Warning}
\newtheorem{sch}[thm]{Scholium}

\newenvironment{myitemize}
{ \begin{itemize}
    \setlength{\itemsep}{4pt}
    \setlength{\parskip}{4pt}
    \setlength{\parsep}{4pt}     }
{ \end{itemize}                  }

\makeatletter
\let\c@equation\c@thm
\raggedbottom
\makeatother
\numberwithin{equation}{section}
%--------Meta Data: Fill in your info------
%&#1043;&#1086;&#1089;&#1087;&#1086;&#1076;&#1080;, &#1048;&#1080;&#1089;&#1091;&#1089;&#1077; &#1061;&#1088;&#1080;&#1089;&#1090;&#1077;, &#1057;&#1099;&#1085;&#1077; &#1041;&#1086;&#1078;&#1080;&#1081;, &#1084;&#1086;&#1083;&#1080;&#1090;&#1074;&#1072;&#1084;&#1080; &#1055;&#1088;&#1077;&#1095;&#1080;&#1089;&#1090;&#1099;&#1103; &#1058;&#1074;&#1086;&#1077;&#1103; &#1052;&#1072;&#1090;&#1077;&#1088;&#1080; &#1080; &#1042;&#1089;&#1077;&#1093; &#1057;&#1074;&#1103;&#1090;&#1099;&#1093; &#1058;&#1074;&#1086;&#1080;&#1093;, &#1087;&#1086;&#1084;&#1080;&#1083;&#1091;&#1081; &#1085;&#1072;&#1089;. &#1040;&#1084;&#1080;&#1085;&#1100;.
\usepackage{microtype}
\begin{document}

\section{Structure of the manuscript}
The content of the manuscript reaches its main aim through the following milestones. Firstly, we perform a detailed discussion on polynomials P, we show all their implicit forms and discuss the main properties. Finally, for the first milestone, we arrive to the identity between odd-powered Binomial (and Multinomial) expansions and partial case of P. As next step, we establish a relation between the polynomials P and discrete convolution of piecewise defined power function. This relation is consequence of the following claims
\begin{itemize}
  \item P is in relation with the power sum Q.
  \item Discrete convolution of piecewise defined power function is partial case of the power sum Q.
  \item P is in relation with discrete convolution.
\end{itemize}
Then, the subsection [name] particularise obtained results to show the relation between Binomial (and Multinomial) theorem and the discrete convolution of piecewise defined power function. After, the straightforward section named [name] discuss the finite differences and derivatives of power function applying an identity between binomial theorem and convolution. 
\end{document}
