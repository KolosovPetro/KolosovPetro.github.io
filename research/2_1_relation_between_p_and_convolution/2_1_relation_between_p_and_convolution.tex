\documentclass[12pt, letterpaper]{amsart}
\usepackage[left=1in,right=1in,bottom=1in,top=1in]{geometry}
\usepackage{amsfonts}
\usepackage{amsmath, amssymb}
\usepackage[font=small,labelfont=bf]{caption}
\usepackage[pdfpagelabels,hyperindex,colorlinks=true,linkcolor=blue,urlcolor=magenta,citecolor=green]{hyperref}
\usepackage{amsthm}
\usepackage{float}
\usepackage{mathrsfs}
\usepackage{colonequals}
\usepackage{natbib}

\hypersetup{
pdftitle={Manuscript title},
pdfsubject={Mathematics},
pdfauthor={Petro Kolosov},
pdfkeywords={}
}


\newtheorem{thm}{Theorem}[section]
\newtheorem{cor}[thm]{Corollary}
\newtheorem{prop}[thm]{Proposition}
\newtheorem{lem}[thm]{Lemma}
\newtheorem{conj}[thm]{Conjecture}
\newtheorem{quest}[thm]{Question}
\newtheorem{ppty}[thm]{Property}
\newtheorem{ppties}[thm]{Properties}
\newtheorem{claim}[thm]{Claim}

\theoremstyle{definition}
\newtheorem{defn}[thm]{Definition}
\newtheorem{defns}[thm]{Definitions}
\newtheorem{con}[thm]{Construction}
\newtheorem{exmp}[thm]{Example}
\newtheorem{exmps}[thm]{Examples}
\newtheorem{notn}[thm]{Notation}
\newtheorem{notns}[thm]{Notations}
\newtheorem{addm}[thm]{Addendum}
\newtheorem{exer}[thm]{Exercise}
\newtheorem{limit}[thm]{Limitation}

\theoremstyle{remark}
\newtheorem{rem}[thm]{Remark}
\newtheorem{rems}[thm]{Remarks}
\newtheorem{warn}[thm]{Warning}
\newtheorem{sch}[thm]{Scholium}

\makeatletter
\let\c@equation\c@thm
\raggedbottom
\makeatother
\numberwithin{equation}{section}
%--------Meta Data: Fill in your info------
%&#1043;&#1086;&#1089;&#1087;&#1086;&#1076;&#1080;, &#1048;&#1080;&#1089;&#1091;&#1089;&#1077; &#1061;&#1088;&#1080;&#1089;&#1090;&#1077;, &#1057;&#1099;&#1085;&#1077; &#1041;&#1086;&#1078;&#1080;&#1081;, &#1084;&#1086;&#1083;&#1080;&#1090;&#1074;&#1072;&#1084;&#1080; &#1055;&#1088;&#1077;&#1095;&#1080;&#1089;&#1090;&#1099;&#1103; &#1058;&#1074;&#1086;&#1077;&#1103; &#1052;&#1072;&#1090;&#1077;&#1088;&#1080; &#1080; &#1042;&#1089;&#1077;&#1093; &#1057;&#1074;&#1103;&#1090;&#1099;&#1093; &#1058;&#1074;&#1086;&#1080;&#1093;, &#1087;&#1086;&#1084;&#1080;&#1083;&#1091;&#1081; &#1085;&#1072;&#1089;. &#1040;&#1084;&#1080;&#1085;&#1100;.
\usepackage{microtype}
\begin{document}
\section{Relation between P and Convolution}
Previously we have established a relation between polynomial P and Binomial, Multinomial theorems. In this section a relation between P and convolution of the piecewise defined power function $f_{t}^{r}$ is established. To show that P implicitly involves the discrete convolution of piecewise defined power function $f_{t}^{r}$ let's refresh what P are
\begin{equation*}
P=\sum \mathbf{A Q}.
\end{equation*}
Meanwhile, the term Q is the power sum of the form
\begin{equation*}
\mathbf{Q} = \sum k^r (n-k)^r
\end{equation*}
It could be noticed immediately that Q differs from the discrete convolution of $f_{t}^{r}$ only in sense of boundary conditions of the summation. For instance, the discrete convolution of the piecewise defined power function $f_{t}^{r}$ is
\begin{equation*}
\begin{split}
(f_{t}^{r} \ast f_{t}^{r})[n]
&= \sum_{k}f_{r,t}(k)f_{r,t}(n-k) = \sum_{k}k^r(n-k)^r[k\geq t][n-k\geq t] \\
&= \sum_{k}k^r(n-k)^r[t\leq k \leq n-t].
\end{split}
\end{equation*}
It is now clear that discrete convolution $(f_{t}^{k} \ast f_{t}^{k})[n]$ of piecewise defined power function $f_{t}^{r}$ is a partial case of the power sum Q with $a=$ and $b=$, ie
\begin{equation*}
(f_{t}^{r} \ast f_{t}^{r})[n]  = \mathbf{Q}_{t,n-t+1}^r(n), \quad n\geq 1.
\end{equation*}
Therefore, the polynomials $\mathbf{P}^{m}_{a,b}(n)$ are in relation with discrete convolution of piecewise defined power function $f_{t}^{r}$ as follows
\begin{equation*}
\mathbf{P}^{m}_{t,n-t+1}(n)
=\sum\limits_{r}\mathbf{A}_{m,r} \mathbf{Q}_{t,n-t+1}^r(n)
\equiv \sum\limits_{r}\mathbf{A}_{m,r} (f_{t}^{r} \ast f_{t}^{r})[n], \quad n\geq 1.
\end{equation*}
Following this logic, we are able to find a relation between $P$ and discrete convolution of $f_{t}^{r}$.
\subsection{Relation between Binomial theorem and Convolution}
As it is stated previously in (exp link), the polynomials P are able to be expressed in terms of convolution $(f_{r,t} \ast f_{r,t})[n]$ of $f_{r,t}(n)$. Consequently, by the equivalence between BT and P which is (4), the Binomial expansion could be expressed in terms of convolution $n_{\geq t}^r \ast n_{\geq t}^r$ as well,
\begin{equation*}
\begin{split}
(a+b)^{2m+1}=\sum_{r} \binom{2m+1}{r} a^{2m+1-r} b^r
&\equiv -1+\mathbf{P}^{m}_{a+b+1}(a+b) \\
&= -1+\sum_{r}\mathbf{A}_{m,r}\mathbf{Q}^{r}_{a+b+1}(a+b)\\
&= -1+\sum\limits_{r}\mathbf{A}_{m,r} (f^{r} \ast f^{r})[a+b].
\end{split}
\end{equation*}
\end{document}
